
\documentclass[11pt, a4paper]{article}

\usepackage[top=1in, bottom=1in, left=1in, right=1in]{geometry}
%\usepackage{setspace}
%\onehalfspacing
\usepackage{graphicx}
\usepackage{float}

\usepackage{subfig}
\usepackage{amsmath}

\begin{document}

\title{Hist380 Notes}
\author{Jay Mundrawala}
\date{\today}
\maketitle

\section{The Republic, 509-27 BCE}
\begin{description}
  \item[patricians] the senate - chief execs - consuls
  \item[plebians] the Assembly (Comitia Curiata)
\end{description}

gradually plebians get rights. Why?
\begin{itemize}
  \item Comitia Tribute 494 BCE
    \begin{itemize}
      \item tribunes
      \item veto power
    \end{itemize}
  \item 12 Tables of law, 450 BCE
  \item rich plebians can marry patricians, 445 BCE
  \item patrician consul, 1 plebian consul - 300s BCE
\end{itemize}

By 264 BCE

2 consuls

Senate - 300
\begin{itemize}
  \item foreign policy
  \item advisory capacity to magistrates
  \item for life
\end{itemize}

3 Assemblies
\begin{itemize}
  \item 10 tribunes
\end{itemize}

2 censors
\begin{itemize}
  \item 5 yr term
  \item Maraus Porcius Cato ``the Elder'' (234-149 BCE)
\end{itemize}

Pontifex Maximus

Dictator
\begin{itemize}
  \item Lucius Quinctilus Cincinnatus
\end{itemize}

patron-client system

\section{Roman Expansion}
from the start they're on the move
\begin{itemize}
  \item topography
  \item defensive imperialism; Gallic invasion, 387 BCE
  \item dogged persistence
\end{itemize}

\section{Italy, 509-264 BCE}
Latin Status

Allied Status

Municipal status

\section{Western Mediterranean}
Punic Wars, 264 - 146 BCE

Poenus = Phoenician

First Punic War, 264 BCE-241 BCE

\end{document}
