
\documentclass[10pt, a4paper]{article}

\usepackage[top=1in, bottom=1in, left=1in, right=1in]{geometry}
%\usepackage{setspace}
%\onehalfspacing
\usepackage{graphicx}
\usepackage{float}

\usepackage{subfig}
\usepackage{amsmath}

\usepackage{amssymb}
\usepackage{fancyhdr}
\usepackage{listings}
\usepackage{textcomp}
\usepackage{upquote}
\pagestyle{fancyplain}

\renewcommand{\arraystretch}{1.5}

\begin{document}
\lhead{Jay Mundrawala}
\rhead{ECE 481 - Homework 1}

\section{Computer Assignment}
Figure \ref{fig:fft} shows the fourier transform for the original image.
\linebreak
\begin{figure}[h!]
  \centering
  \includegraphics{../data/fft2.eps}
  \caption{Results for filtering}
  \label{fig:fft}
\end{figure}

Figure \ref{fig:filter} shows the results for applying the filter for different values of $\sigma$. The best result 
comes from either $\sigma = 40$ or $\sigma = 50$. Lower values start bluring the image, while higher values do not
remove the stripes.
\begin{figure}[h!]
  \centering
  \includegraphics{../data/plot2.eps}
  \caption{Results for filtering}
  \label{fig:filter}
\end{figure}
This can be seen in Figure \ref{fig:intensity}. This image shows the intensity of the two stars. As sigma goes down, so does
their intensity. After a certain point, they are just blured together and not only can they not be distunguished from each other,
but it is very difficult to distinguish the blob of stars from the background.

\begin{figure}[h!]
  \centering
  \includegraphics{../data/intensity_2.eps}
  \caption{Intensity Plots of 2 star section}
  \label{fig:intensity}
\end{figure}

\end{document}

