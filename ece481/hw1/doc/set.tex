
\documentclass[10pt, a4paper]{article}

\usepackage[top=1in, bottom=1in, left=1in, right=1in]{geometry}
%\usepackage{setspace}
%\onehalfspacing
\usepackage{graphicx}
\usepackage{float}

\usepackage{subfig}
\usepackage{amsmath}

\usepackage{amssymb}
\usepackage{fancyhdr}
\pagestyle{fancyplain}

\renewcommand{\arraystretch}{1.5}

\begin{document}
\lhead{Jay Mundrawala}
\rhead{ECE 481 - Homework 1}

\begin{enumerate}
  \item[1a. ]
        \begin{equation} \label{eq:womix}
          \alpha_i(C) = \int C(\lambda) S_{i}(\lambda)\, d\lambda \\
        \end{equation}
        \begin{equation} \label{eq:wmix}
          \alpha_i(M) = \int \underbrace{\left[\sum_{k=1}^{2}\beta_{k}(C)P_{k}(\lambda) \right]}_{M(\lambda)}S_i(\lambda)\, d\lambda \\
        \end{equation}
        From \eqref{eq:womix}:
        \begin{eqnarray}
          \alpha_1(C) &=& \int^{6}_{5} \left( \frac{1}{2}\lambda -2 \right) \, d\lambda \\
          &=& \left[\frac{1}{4}\lambda^{2} - 2\lambda \right]^{6}_{5} + \left[\lambda \right]^{7}_{6} \\
          &=& \dfrac{3}{4} + 1 \\
          &=& \dfrac{7}{4}
        \end{eqnarray}
        \begin{eqnarray}
          \alpha_2(C) &=& \int^6_5 \left(\frac{1}{2}\right) \, d\lambda \\
          &=& \dfrac{1}{2}
        \end{eqnarray}
        If
        $P_1(\lambda) = \delta (\lambda-5)$
        and
        $P_2(\lambda) = \delta (\lambda-7)$
        then, from \eqref{eq:wmix}
        \begin{eqnarray}
          \alpha_1(M) &=& \int^6_4 \left[\beta_1\delta(\lambda-5) + \beta_2\delta(\lambda-7)\right]\left(\frac{1}{2}\lambda-2\right)\,
          d\lambda \\
          && + \int^8_6\left[\beta_1\delta(\lambda-5) + \beta_2\delta(\lambda-7)\right]\,d\lambda \nonumber \\
          &=& \left[\left(\dfrac{1}{2}\right)\beta_1 + \left(\dfrac{3}{2}\right)\beta_2\right] + \left[\beta_1 + \beta_2 \right] \\
          &=& \dfrac{3}{2}\beta_1 + \dfrac{5}{2}\beta_2 = \dfrac{7}{4}
        \end{eqnarray}
        \begin{eqnarray}
          \alpha_2(M) &=& \int^6_4\left[\beta_1\delta(\lambda-5) + \beta_2\delta(\lambda-7)\right]\left(\dfrac{1}{2}\right)\,d\lambda \\
          &=& \dfrac{1}{2}\beta_1 + \dfrac{1}{2}\beta_2 = \dfrac{1}{2}
        \end{eqnarray}
        Solving for $\beta_1$ and $\beta_2$
        \begin{eqnarray}
          \dfrac{3}{2}\beta_1 + \dfrac{5}{2}\beta_2 &=& \dfrac{7}{4} \\  
          \beta_1 + \beta_2 &=& 1 \\
          \beta_1 &=& \dfrac{3}{4} \\
          \beta_2 &=& \dfrac{1}{4}
        \end{eqnarray}
\end{enumerate}

\end{document}

