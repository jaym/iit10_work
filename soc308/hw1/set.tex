
\documentclass[10pt, a4paper]{article}

\usepackage[top=1in, bottom=1in, left=1in, right=1in]{geometry}
%\usepackage{setspace}
%\onehalfspacing
\usepackage{graphicx}
\usepackage{float}

\usepackage{subfig}
\usepackage{amsmath}

\usepackage{amssymb}
\usepackage{fancyhdr}
\pagestyle{fancyplain}

\renewcommand{\arraystretch}{1.5}

\begin{document}
\lhead{Jay Mundrawala}
\rhead{SOC308 Homework 1}

\begin{enumerate}
  \item Make a list of the different episodes in the Randi tape, and for each, list at least 3 biases that seem to be present for each episode. Briefly explain, for each episode, how these biases are supposedly working here.
    \begin{itemize}
      \item Uri Geller
        \begin{description}
          \item[Preconception] The fact that people believed Uri Geller's claims that his abilities were indeed supernatural led them no not question whether someone really needed supernatural powers to do the tricks he did.
          \item[Behavioral confirmation] The attention he got as a psychic made them treat him like one. I think it was because of this reason that people still believed he had psychic abilities even after the Tonight Show airing. It allowed him to get away with the excuse that he just was not feeling strong enough to perform the tricks he was asked.
        \end{description}
      \item Peter Popoff
        \begin{description}
          \item[Preconception] The people who attended the shows seemed to be true believers that Popoff was divine. They came with this preconception, so anything he did would just cement this idea more. When he stated detailed information about people, those people didn't think to question where he got that information from, as they had written it down for him.
          \item[Confirmation bias] As believers had the preconception that Popoff had the ability to heal people, they didn't question it. So when he was able to make people get up from their wheelchair, people didn't seem to question whether that person could already walk, or that maybe he hired actors, etc.
          \item[Regression to the average] People get sick, then recover naturally. Its possible people could confuse that which Popoff did with this.
        \end{description}
      \item Horoscopes
        \begin{description}
          \item[Illusion of control] Astrology gives people the idea that they can know whats coming, which is very attractive to people. This leads them to believe the things they are told.
          \item[Preconception] Some people have a preconception that astrology works. The fact is that its so vague that these people can twist whatever is written to fit their lives.
          \item[Self-fulfilling prophecy] Its possible that people can make the horoscopes self-fulfilling.
          \item[Hindsight bias] If people believe that its possible to see things before they happen, they would have no problem believing horoscopes.
        \end{description}
      \item Palm reader
        \begin{description}
          \item[Illusion of control] Astrology gives people the idea that they can know whats coming, which is very attractive to people. This leads them to believe the things they are told.
          \item[Preconception] Some people have a preconception that astrology works. The fact is that its so vague that these people can twist whatever is written to fit their lives. The reader actually did the opposite of what he was supposed to and the person still believed it, confirming that people will take whatever is said and try to fit that into their lives.
          \item[Self-fulfilling prophecy] Its possible that people can make the reading self-fulfilling.
        \end{description}
      \item Institute of the Brain
        \begin{description}
          \item[Illusory Correlation] The scientists didn't have a blinded tests, so they were looking for certain results. When Randi asked them to check 4 results for any abnormal activity, they got 1 correct. Randi's test was a blinded one, and $1/4$ is the expected number of correct results if you choose randomly.
          \item[Confirmation bias] Since the scientists were looking for results, anything they found that confirmed the results was used, and that which didn't was ignored. 
        \end{description}
      \item People's Health clinic
        \begin{description}
          \item[Overconfidence Bias] The ``doctors'' were much too overconfident in their water. They claimed it had healing powers, but there was no way to distinguish it from regular water, which made it difficult to scientifically test.
          \item[Regression to the average] People get sick, then recover naturally. If there's no way to distinguish the regular water from the healing water, theres no way to say that it was the healing water the cured a person.
        \end{description}
    \end{itemize}

  \item What biases in his audience could one say that Randi himself is playing with in his presentation of this program?

    I think the main bias' here are preconception and behavioral confirmation. I say preconception because although he says that he cant disprove psychic, he has the notion that they are all frauds. This leads directly to behavioral confirmation. Since he believes that these people are frauds, he treats them like frauds, and thats what they'll come out to be. He's obviously right to do so, but it is a bias nonetheless. 


\end{enumerate}
\end{document}

