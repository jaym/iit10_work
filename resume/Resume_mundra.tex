%%%%%%%%%%%%%%%%%%%%%%%%%%%%%%%%%%%%%%%%%%%%%%%%%%%%%%%%%%%%%%%%%%%%%%%%
%%%%%%%%%%%%%%%%%%%%%% Simple LaTeX CV Template %%%%%%%%%%%%%%%%%%%%%%%%
%%%%%%%%%%%%%%%%%%%%%%%%%%%%%%%%%%%%%%%%%%%%%%%%%%%%%%%%%%%%%%%%%%%%%%%%

%%%%%%%%%%%%%%%%%%%%%%%%%%%%%%%%%%%%%%%%%%%%%%%%%%%%%%%%%%%%%%%%%%%%%%%%
%% NOTE: If you find that it says                                     %%
%%                                                                    %%
%%                           1 of ??                                  %%
%%                                                                    %%
%% at the bottom of your first page, this means that the AUX file     %%
%% was not available when you ran LaTeX on this source. Simply RERUN  %% 
%% LaTeX to get the ``??'' replaced with the number of the last page  %% 
%% of the document. The AUX file will be generated on the first run   %%
%% of LaTeX and used on the second run to fill in all of the          %%
%% references.                                                        %%
%%%%%%%%%%%%%%%%%%%%%%%%%%%%%%%%%%%%%%%%%%%%%%%%%%%%%%%%%%%%%%%%%%%%%%%%

%%%%%%%%%%%%%%%%%%%%%%%%%%%% Document Setup %%%%%%%%%%%%%%%%%%%%%%%%%%%%

% Don't like 10pt? Try 11pt or 12pt
\documentclass[10pt]{article}

% This is a helpful package that puts math inside length specifications
\usepackage{calc}

% Layout: Puts the section titles on left side of page
\reversemarginpar

%
%         PAPER SIZE, PAGE NUMBER, AND DOCUMENT LAYOUT NOTES:
%
% The next \usepackage line changes the layout for CV style section
% headings as marginal notes. It also sets up the paper size as either
% letter or A4. By default, letter was used. If A4 paper is desired,
% comment out the letterpaper lines and uncomment the a4paper lines.
%
% As you can see, the margin widths and section title widths can be
% easily adjusted.
%
% ALSO: Notice that the includefoot option can be commented OUT in order
% to put the PAGE NUMBER *IN* the bottom margin. This will make the
% effective text area larger.
%
% IF YOU WISH TO REMOVE THE ``of LASTPAGE'' next to each page number,
% see the note about the +LP and -LP lines below. Comment out the +LP
% and uncomment the -LP.
%
% IF YOU WISH TO REMOVE PAGE NUMBERS, be sure that the includefoot line
% is uncommented and ALSO uncomment the \pagestyle{empty} a few lines
% below.
%

%% Use these lines for letter-sized paper
\usepackage[paper=letterpaper,
            %includefoot, % Uncomment to put page number above margin
            marginparwidth=1.2in,     % Length of section titles
            marginparsep=.05in,       % Space between titles and text
            margin=1in,               % 1 inch margins
            includemp]{geometry}

%% Use these lines for A4-sized paper
%\usepackage[paper=a4paper,
%            %includefoot, % Uncomment to put page number above margin
%            marginparwidth=30.5mm,    % Length of section titles
%            marginparsep=1.5mm,       % Space between titles and text
%            margin=25mm,              % 25mm margins
%            includemp]{geometry}

%% More layout: Get rid of indenting throughout entire document
\setlength{\parindent}{0in}

%% This gives us fun enumeration environments. compactenum will be nice.
\usepackage{paralist}

%% Reference the last page in the page number
%
% NOTE: comment the +LP line and uncomment the -LP line to have page
%       numbers without the ``of ##'' last page reference)
%
% NOTE: uncomment the \pagestyle{empty} line to get rid of all page
%       numbers (make sure includefoot is commented out above)
%
\usepackage{fancyhdr,lastpage}
\pagestyle{fancy}
\pagestyle{empty}      % Uncomment this to get rid of page numbers
\fancyhf{}\renewcommand{\headrulewidth}{0pt}
\fancyfootoffset{\marginparsep+\marginparwidth}
\newlength{\footpageshift}
\setlength{\footpageshift}
          {0.5\textwidth+0.5\marginparsep+0.5\marginparwidth-2in}
\lfoot{\hspace{\footpageshift}%
       \parbox{4in}{\, \hfill %
                    \arabic{page} of \protect\pageref*{LastPage} % +LP
%                    \arabic{page}                               % -LP
                    \hfill \,}}

% Finally, give us PDF bookmarks
\usepackage{color,hyperref}
\definecolor{darkblue}{rgb}{0.0,0.0,0.3}
\hypersetup{colorlinks,breaklinks,
            linkcolor=darkblue,urlcolor=darkblue,
            anchorcolor=darkblue,citecolor=darkblue}

%%%%%%%%%%%%%%%%%%%%%%%% End Document Setup %%%%%%%%%%%%%%%%%%%%%%%%%%%%


%%%%%%%%%%%%%%%%%%%%%%%%%%% Helper Commands %%%%%%%%%%%%%%%%%%%%%%%%%%%%

% The title (name) with a horizontal rule under it
%
% Usage: \makeheading{name}
%
% Place at top of document. It should be the first thing.
\newcommand{\makeheading}[1]%
        {\hspace*{-\marginparsep minus \marginparwidth}%
         \begin{minipage}[t]{\textwidth+\marginparwidth+\marginparsep}%
                {\large \bfseries #1}\\[-0.15\baselineskip]%
                 \rule{\columnwidth}{1pt}%
         \end{minipage}}

% The section headings
%
% Usage: \section{section name}
%
% Follow this section IMMEDIATELY with the first line of the section
% text. Do not put whitespace in between. That is, do this:
%
%       \section{My Information}
%       Here is my information.
%
% and NOT this:
%
%       \section{My Information}
%
%       Here is my information.
%
% Otherwise the top of the section header will not line up with the top
% of the section. Of course, using a single comment character (%) on
% empty lines allows for the function of the first example with the
% readability of the second example.
\renewcommand{\section}[2]%
        {\pagebreak[2]\vspace{1.3\baselineskip}%
         \phantomsection\addcontentsline{toc}{section}{#1}%
         \hspace{0in}%
         \marginpar{
         \raggedright \scshape #1}#2}

% An itemize-style list with lots of space between items
\newenvironment{outerlist}[1][\enskip\textbullet]%
        {\begin{enumerate}[#1]}{\end{enumerate}%
         \vspace{-.6\baselineskip}}

% An itemize-style list with little space between items
\newenvironment{innerlist}[1][\enskip\textbullet]%
        {\begin{compactenum}[#1]}{\end{compactenum}}

% To add some paragraph space between lines.
% This also tells LaTeX to preferably break a page on one of these gaps
% if there is a needed pagebreak nearby.
\newcommand{\blankline}{\quad\pagebreak[2]}

%%%%%%%%%%%%%%%%%%%%%%%% End Helper Commands %%%%%%%%%%%%%%%%%%%%%%%%%%%

%%%%%%%%%%%%%%%%%%%%%%%%% Begin CV Document %%%%%%%%%%%%%%%%%%%%%%%%%%%%

\begin{document}
\makeheading{Jay Dave Mundrawala \hfill \tiny{Revised: January 2010}}

\section{Contact Information}
%
% NOTE: Mind where the & separators and \\ breaks are in the following
%       table.
%
% ALSO: \rcollength is the width of the right column of the table 
%       (adjust it to your liking; default is 1.85in).
%
\newlength{\rcollength}\setlength{\rcollength}{1.85in}%
%
\begin{tabular}[t]{@{}p{.67\textwidth}lp{.17\textwidth}lp{.2\textwidth}}
Jay D. Mundrawala       &	\textit{Phone:} & (630) 926-6759\\
1208 Cromwell Lane      &	\textit{Fax:} & (630) 904-0169\\
Naperville, IL  60564 	&	\textit{E-mail:} & jay@ir.iit.edu\\

\end{tabular}

\section{Professional Interests}
%
Data Mining, Information Retrieval, Hardware Design, High Performance Computing

\section{Education}
%
\textbf{Illinois Institute of Technology},
Chicago, IL
\begin{outerlist}
	\item[] 
        \begin{innerlist}
        \item B.S., Computer Engineering \hfill \textbf{Aug. 2007 to May 2010}
        \end{innerlist}
\end{outerlist}

%\blankline

%\textbf{Saint Patrick High School}, 
%Chicago, IL
%\begin{outerlist}
%	\item[]
%	\begin{innerlist}
%	\item Graduated 2005: 4 year B+ Honor Roll student
%	\end{innerlist}
%\end{outerlist}

\section{Publication} % (fold)
\label{sec:publications}
N. Goharian, O. Frieder, W. Yee, J. Mundrawala, ``Enriching Peer-to-Peer File Descriptors using Association Rules on Query Logs,'' 32nd European Conference on Information Retrieval, Milton Keys, UK, March 2010.
% section publications (end)

\section{Technical\\ Skills}
Focus in information retrieval and digital hardware design
\begin{outerlist}
	\item[] Programming Languages
	\begin{innerlist}
		\item \textit{Proficient:}  ASM(MIPS/M68K), C, Java, VHDL
		\item \textit{Familiar:}  BASH, Erlang, GNU flex, Perl, Ruby, SQL
	\end{innerlist}
	\item[] Operating Systems:  GNU/Linux, Windows
	\item[] Tools: Apache Ant, Git, GLib, \LaTeX, Lucene, Make, ModelSim, MySQL, OpenCL, SQLite, Vim, Xilinx ISE 
\end{outerlist}

\section{Professional Experience}
\textbf{Information Retrieval Laboratory} - Illinois Institute of Technology, Chicago, IL
\begin{outerlist}
	\item[] \textit{Research Assistant} \hfill \textbf{May. 2008 to present}
	\begin{innerlist}
		\item Debug and optimize SQLGenerator\footnote{http://ir.iit.edu/projects/SQLGenerator.html} - A scalable XML retrieval engine which implements XPath and XQuery by translating it to SQL.
        \item Update documentation for various projects
		\item Run test to determine efficiency of selected search engines
        \item Design and Implement an encryption solution for encrypting a Lucene index
	\end{innerlist}
\end{outerlist}

\blankline

\textbf{Defyned, Inc.}
\begin{outerlist}
	\item[] \textit{Senior Developer} \hfill \textbf{Jul. 2009 to Oct. 2009}
	\begin{innerlist}
		\item building a massively scalable chat system
        \item integrating ability to perform topic detection on chats
	\end{innerlist}
\end{outerlist}

\blankline

\textbf{Computer Science Department} - Illinois Institute of Technology, Chicago, IL
\begin{outerlist}
	\item[] \textit{Teaching Assistant} \hfill \textbf{Jan. 2008 to May. 2008}
	\begin{innerlist}
        \item Assist students with various C programming assignments
	\end{innerlist}
\end{outerlist}


%\section{References}
%\textit{Available upon request}

\end{document}


%%%%%%%%%%%%%%%%%%%%%%%%%% End CV Document %%%%%%%%%%%%%%%%%%%%%%%%%%%%%
